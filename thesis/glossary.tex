\newglossaryentry{api}
{
  name=\glslink{API}{API},
  text= Application Programming Interface,
  sort= api,
  description= {The set of defined rules that enable different applications to communicate with each other}
}
\newacronym[description={\glslink{api}{Application Programming Interface}}]{API}{API}{Application Programming Interface}

\newglossaryentry{http}
{
  name=\glslink{HTTP}{HTTP},
  text= Hypertext Transfer Protocol,
  sort= http,
  description= {The communications protocol used to connect to web servers on the Internet}
}
\newacronym[description={\glslink{http}{Hypertext Transfer Protocol}}]{HTTP}{HTTP}{Hypertext Transfer Protocol}

\newglossaryentry{html}
{
  name=\glslink{HTML}{HTML},
  text= Hypertext Markup Language,
  sort= http,
  description= {The markup language for the web that defines the structure of web pages}
}
\newacronym[description={\glslink{html}{Hypertext Markup Language}}]{HTML}{HTML}{Hypertext Markup Language}

\newglossaryentry{url}
{
  name=\glslink{URL}{URL},
  text= Uniform Resource Locator,
  sort= url,
  description= {It is a unique identifier used to locate a resource on the Internet. It consists of multiple parts, including a protocol and domain name, that tell a web browser how and where to retrieve a resource}
}
\newacronym[description={\glslink{url}{Uniform Resource Locator}}]{URL}{URL}{Uniform Resource Locator}

\newglossaryentry{dom}
{
  name=\glslink{DOM}{DOM},
  text= Document Object Model,
  sort= dom,
  description= {The data representation of the objects that comprise the structure and content of a document on the web}
}
\newacronym[description={\glslink{dom}{Document Object Model}}]{DOM}{DOM}{Document Object Model}

\newglossaryentry{ai}
{
  name=\glslink{AI}{AI},
  text= Artificial Intelligence,
  sort= ai,
  description= {It is the science and engineering of making intelligent machines, especially intelligent computer programs. It is related to the similar task of using computers to understand human intelligence}
}
\newacronym[description={\glslink{ai}{Artificial Intelligence}}]{AI}{AI}{Artificial Intelligence}

\newglossaryentry{it}
{
  name=\glslink{IT}{IT},
  text= Information Technology,
  sort= it,
  description= {It is the use of any computers, storage, networking and other physical devices, infrastructure and processes to create, process, store, secure and exchange all forms of electronic data}
}
\newacronym[description={\glslink{it}{Information Technology}}]{IT}{IT}{Information Technology}

\newglossaryentry{nist}
{
  name=\glslink{NIST}{NIST},
  text= National Institute of Standards and Technology,
  sort= nist,
  description= {It is a United States government agency responsible for developing and promoting measurement standards and technology advancements to enhance innovation and industrial competitiveness}
}
\newacronym[description={\glslink{nist}{National Institute of Standards and Technology}}]{NIST}{NIST}{National Institute of Standards and Technology}

\newglossaryentry{aws}
{
  name=\glslink{AWS}{AWS},
  text= Amazon Web Services,
  sort= aws,
  description= {It is a comprehensive and widely-used cloud computing platform provided by Amazon}
}
\newacronym[description={\glslink{aws}{Amazon Web Services}}]{AWS}{AWS}{Amazon Web Services}

\newglossaryentry{gcp}
{
  name=\glslink{GCP}{GCP},
  text= Google Cloud Platform,
  sort= gcp,
  description= {It is a suite of cloud computing services offered by Google}
}
\newacronym[description={\glslink{gcp}{Google Cloud Platform}}]{GCP}{GCP}{Google Cloud Platform}

\newglossaryentry{gke}
{
  name=\glslink{GKE}{GKE},
  text= Google Kubernetes Engine,
  sort= gke,
  description= {It is a Google-managed implementation of the Kubernetes open source container orchestration platform}
}
\newacronym[description={\glslink{gke}{Google Kubernetes Engine}}]{GKE}{GKE}{Google Kubernetes Engine}

\newglossaryentry{cncf}
{
  name=\glslink{CNCF}{CNCF},
  text= Cloud Native Computing Foundation,
  sort= cncf,
  description= {It is an open-source software foundation that promotes the adoption of cloud-native computing}
}
\newacronym[description={\glslink{cncf}{Cloud Native Computing Foundation}}]{CNCF}{CNCF}{Cloud Native Computing Foundation}

\newglossaryentry{csp}
{
  name=\glslink{CSP}{CSP},
  text= Cloud Service Provider,
  sort= csp,
  description= {It is a third-party company that provides scalable computing resources that businesses can access on demand over a network, including cloud-based computing, storage, platform, and application services}
}
\newacronym[description={\glslink{csp}{Cloud Service Provider}}]{CSP}{CSP}{Cloud Service Provider}

\newglossaryentry{tor}
{
  name=\glslink{TOR}{TOR},
  text= The Onion Router,
  sort= tor,
  description= {It is free and open-source software for enabling anonymous communication and directs Internet traffic via a free, worldwide, volunteer overlay network comprising more than seven thousand relays}
}
\newacronym[description={\glslink{tor}{The Onion Router}}]{TOR}{TOR}{The Onion Router}

\newglossaryentry{www}
{
  name=\glslink{WWW}{WWW},
  text= World Wide Web,
  sort= www,
  description= {It refers to all websites or public pages that users can access from their local computers and other devices via the Internet. These pages and documents are interconnected by means of hyperlinks that users click on to obtain information. This information can be in different formats, including text, images, audio and video}
}
\newacronym[description={\glslink{www}{World Wide Web}}]{WWW}{WWW}{World Wide Web}

\newglossaryentry{wsr}
{
  name=\glslink{WSR}{WSR},
  text= WebGather Server Registry,
  sort= wsr,
  description= {It is a coordinator module within the WebGather system. It serves as a central storage unit for essential information, including the IPs and ports of all registered main controllers in the system}
}
\newacronym[description={\glslink{wsr}{WebGather Server Registry}}]{WSR}{WSR}{WebGather Server Registry}

\newglossaryentry{json}
{
  name=\glslink{JSON}{JSON},
  text= JavaScript Object Notation,
  sort= json,
  description= {It is a standard text-based format for representing structured data based on JavaScript object syntax}
}
\newacronym[description={\glslink{json}{JavaScript Object Notation}}]{JSON}{JSON}{JavaScript Object Notation}

\newglossaryentry{cli}
{
  name=\glslink{CLI}{CLI},
  text= Command Line Interface,
  sort= cli,
  description= {It is a text-based interface where you can input commands that interact with a computer's software}
}
\newacronym[description={\glslink{cli}{Command Line Interface}}]{CLI}{CLI}{Command Line Interface}

\newglossaryentry{spa}
{
  name=\glslink{SPA}{SPA},
  text= Single-page Application,
  sort= spa,
  description= {It is a web application or website that interacts with the user by dynamically rewriting the current web page with new data from the web server, instead of the default method of a web browser loading entire new pages}
}
\newacronym[description={\glslink{spa}{Single-page Application}}]{SPA}{SPA}{Single-page Application}

\newglossaryentry{rest}
{
  name=\glslink{REST}{REST},
  text= Representational State Transfer,
  sort= rest,
  description= {It is an architectural style for designing networked applications. It is based on a stateless, client-server communication protocol, typically the \acrshort{HTTP} protocol and it is commonly used in web services development}
}
\newacronym[description={\glslink{rest}{Representational State Transfer}}]{REST}{REST}{Representational State Transfer}

\newglossaryentry{sspl}
{
  name=\glslink{SSPL}{SSPL},
  text= Server Side Public License,
  sort= sspl,
  description= {It is a source-available software license introduced by MongoDB Inc. in 2018}
}
\newacronym[description={\glslink{sspl}{Server Side Public License}}]{SSPL}{SSPL}{Server Side Public License}

\newglossaryentry{eck}
{
  name=\glslink{ECK}{ECK},
  text= Elastic Cloud on Kubernetes,
  sort= eck,
  description= {It is the official operator by Elastic for automating the deployment, provisioning, management, and orchestration of Elasticsearch, Kibana, APM Server, Beats, Enterprise Search, Elastic Agent, Elastic Maps Server, and Logstash on Kubernetes}
}
\newacronym[description={\glslink{eck}{Elastic Cloud on Kubernetes}}]{ECK}{ECK}{Elastic Cloud on Kubernetes}

\newglossaryentry{oci}
{
  name=\glslink{OCI}{OCI},
  text= Open Container Initiative,
  sort= oci,
  description= {It is an open governance structure for the express purpose of creating open industry standards around container formats and runtimes}
}
\newacronym[description={\glslink{oci}{Open Container Initiative}}]{OCI}{OCI}{Open Container Initiative}

\newglossaryentry{crd}
{
  name=\glslink{CRD}{CRD},
  text= Custom Resource Definition,
  sort= crd,
  description= {It enables the introduction of unique objects or types into Kubernetes clusters to meet the needs of developers}
}
\newacronym[description={\glslink{crd}{Custom Resource Definition}}]{CRD}{CRD}{Custom Resource Definition}

\newglossaryentry{kpa}
{
  name=\glslink{KPA}{KPA},
  text= Knative Pod Autoscaler,
  sort= kpa,
  description= {The default automatic scaling method inside Knative Serving. It allows the scale to zero and it works with different metrics}
}
\newacronym[description={\glslink{kpa}{Knative Pod Autoscaler}}]{KPA}{KPA}{Knative Pod Autoscaler}

\newglossaryentry{hpa}
{
  name=\glslink{HPA}{HPA},
  text= Horizontal Pod Autoscaler,
  sort= hpa,
  description= {The default scaling method in Kubernetes cluster}
}
\newacronym[description={\glslink{hpa}{Horizontal Pod Autoscaler}}]{HPA}{HPA}{Horizontal Pod Autoscaler}

\newglossaryentry{sdk}
{
  name=\glslink{SDK}{SDK},
  text= Software Development Kit,
  sort= sdk,
  description= {It is a tool for third-party developers to use in producing applications using a particular framework or platform}
}
\newacronym[description={\glslink{sdk}{Software Development Kit}}]{SDK}{SDK}{Software Development Kit}

\newglossaryentry{sqs}
{
  name=\glslink{SQS}{SQS},
  text= Simple Queue Service,
  sort= sqs,
  description= {It is a fully managed message queuing service that makes it easy to decouple and scale microservices, distributed systems, and serverless applications. It moves data between distributed application components and helps you decouple these components}
}
\newacronym[description={\glslink{sqs}{Simple Queue Service}}]{SQS}{SQS}{Simple Queue Service}

\newglossaryentry{sns}
{
  name=\glslink{SNS}{SNS},
  text= Simple Notification Service,
  sort= sns,
  description= {It is a managed service that provides message delivery from publishers to subscribers. Publishers communicate asynchronously with subscribers by sending messages to a topic, which is a logical access point and communication channel. Clients can subscribe to the SNS topic and receive published messages using a supported endpoint type, such as Amazon Kinesis Data Firehose, Amazon SQS, AWS Lambda, HTTP, email, mobile push notifications, and mobile text messages (SMS)}
}
\newacronym[description={\glslink{sns}{Simple Notification Service}}]{SNS}{SNS}{Simple Notification Service}

\newglossaryentry{s3}
{
  name=\glslink{S3}{S3},
  text= Simple Storage Service,
  sort= s3,
  description= {It is an object storage service provided by Amazon that offers industry-leading scalability, data availability, security, and performance}
}
\newacronym[description={\glslink{s3}{Simple Storage Service}}]{S3}{S3}{Simple Storage Service}

\newglossaryentry{ui}
{
  name=\glslink{UI}{UI},
  text= User Interface,
  sort= ui,
  description= {It is the point of human-computer interaction and communication in a device. This can include display screens, keyboards, a mouse and the appearance of a desktop}
}
\newacronym[description={\glslink{ui}{User Interface}}]{UI}{UI}{User Interface}

\newglossaryentry{ec2}
{
  name=\glslink{EC2}{EC2},
  text= Elastic Compute Cloud,
  sort= ec2,
  description= {It provides on-demand, scalable computing capacity in the Amazon Web Services Cloud}
}
\newacronym[description={\glslink{ec2}{Elastic Compute Cloud}}]{EC2}{EC2}{Elastic Compute Cloud}

\newglossaryentry{ecs}
{
  name=\glslink{ECS}{ECS},
  text= Elastic Container Service,
  sort= ecs,
  description= {It is a fully managed container orchestration service that helps you easily deploy, manage, and scale containerized applications}
}
\newacronym[description={\glslink{ecs}{Elastic Container Service}}]{ECS}{ECS}{Elastic Container Service}

\newglossaryentry{elk}
{
  name=\glslink{ELK}{ELK},
  text= Elasticsearch Logstash Kibana,
  sort= elk,
  description= {It is a stack that comprises three popular projects: Elasticsearch, Logstash, and Kibana. Often referred to as Elasticsearch, the ELK stack gives you the ability to aggregate logs from all your systems and applications, analyze these logs, and create visualizations for application and infrastructure monitoring, faster troubleshooting, security analytics, and more}
}
\newacronym[description={\glslink{elk}{Elasticsearch Logstash Kibana}}]{ELK}{ELK}{Elasticsearch Logstash Kibana}

\newglossaryentry{yaml}
{
  name=\glslink{YAML}{YAML},
  text= Yet Another Markup Language,
  sort= yaml,
  description= {It is a human-readable data serialization language that is often used for writing configuration files}
}
\newacronym[description={\glslink{yaml}{Yet Another Markup Language}}]{YAML}{YAML}{Yet Another Markup Language}

\newglossaryentry{ip}
{
  name=\glslink{IP}{IP},
  text= Internet Protocol,
  sort= ip,
  description= {It is the unique identifying number assigned to every device connected to the internet. An IP address definition is a numeric label assigned to devices that use the internet to communicate}
}
\newacronym[description={\glslink{ip}{Internet Protocol}}]{IP}{IP}{Internet Protocol}

\newglossaryentry{dns}
{
  name=\glslink{DNS}{DNS},
  text= Domain Name System,
  sort= dns,
  description= {It translates human-readable domain names to machine-readable IP addresses}
}
\newacronym[description={\glslink{dns}{Domain Name System}}]{DNS}{DNS}{Domain Name System}

\newglossaryentry{uri}
{
  name=\glslink{URI}{URI},
  text= Uniform Resource Identifier,
  sort= uri,
  description= {It is a character sequence that identifies a logical or physical resource, usually connected to the internet. It distinguishes one resource from another}
}
\newacronym[description={\glslink{uri}{Uniform Resource Identifier}}]{URI}{URI}{Uniform Resource Identifier}

\newglossaryentry{mime}
{
  name=\glslink{MIME}{MIME},
  text= Multipurpose Internet Mail Extensions,
  sort= mime,
  description= {It indicates the nature and format of a document, file, or assortment of bytes. See \href{https://developer.mozilla.org/en-US/docs/Web/HTTP/Basics_of_HTTP/MIME_types}{MIME types} for more details}
}
\newacronym[description={\glslink{mime}{Multipurpose Internet Mail Extensions}}]{MIME}{MIME}{Multipurpose Internet Mail Extensions}

\newglossaryentry{ascii}
{
  name=\glslink{ASCII}{ASCII},
  text= American Standard Code for Information Interchange,
  sort= ascii,
  description= {It is the most common character encoding format for text data in computers and on the internet}
}
\newacronym[description={\glslink{ascii}{American Standard Code for Information Interchange}}]{ASCII}{ASCII}{American Standard Code for Information Interchange}

\newglossaryentry{oauth}
{
  name=\glslink{OAuth 2.0}{OAuth 2.0},
  text= Open Authorization,
  sort= oauth,
  description= {It is a standard designed to allow a website or application to access resources hosted by other web apps on behalf of a user}
}
\newacronym[description={\glslink{oauth}{Open Authorization}}]{OAuth 2.0}{OAuth 2.0}{Open Authorization}

\newglossaryentry{crud}
{
  name=\glslink{CRUD}{CRUD},
  text= {CREATE, READ, UPDATE and DELETE},
  sort= crud,
  description= {It describes the four essential operations for creating and managing persistent data elements, mainly in relational and NoSQL databases}
}
\newacronym[description={\glslink{crud}{CREATE, READ, UPDATE and DELETE}}]{CRUD}{CRUD}{CREATE, READ, UPDATE and DELETE}

\newglossaryentry{acm}
{
  name=\glslink{ACM}{ACM},
  text= AWS Certificate Manager,
  sort= acm,
  description= {It is a service that handles the complexity of creating, storing, and renewing public and private SSL/TLS X.509 certificates and keys that protect your AWS websites and applications}
}
\newacronym[description={\glslink{acm}{AWS Certificate Manager}}]{ACM}{ACM}{AWS Certificate Manager}

\newglossaryentry{osint}
{
  name=\glslink{OSINT}{OSINT},
  text= Open Source Intelligence,
  sort= osint,
  description= {It is the act of gathering and analyzing publicly available data for intelligence purposes}
}
\newacronym[description={\glslink{osint}{Open Source Intelligence}}]{OSINT}{OSINT}{Open Source Intelligence}

\newglossaryentry{rbac}
{
  name=\glslink{RBAC}{RBAC},
  text= Role-based access control,
  sort= rbac,
  description= {It is a method of regulating access to computer or network resources based on the roles of individual users within your organization}
}
\newacronym[description={\glslink{rbac}{Role-based access control}}]{RBAC}{RBAC}{Role-based access control}

\newglossaryentry{apk}
{
  name=\glslink{APK}{APK},
  text= Alpine Package Keeper,
  sort= apk,
  description= {It is the package manager of the Alpine Linux distribution. It is used to manage the system packages and the primary method for installing additional software}
}
\newacronym[description={\glslink{apk}{Alpine Package Keeper}}]{APK}{APK}{Alpine Package Keeper}


\newglossaryentry{dls}
{
  name=\glslink{DLS}{DLS},
  text= Dead Letter Sink,
  sort= dls,
  description= {It is a Knative construct that allows the user to configure a destination for events that would otherwise be dropped due to some delivery failure. This is useful for scenarios where you want to ensure that events are not lost due to a failure in the underlying system}
}
\newacronym[description={\glslink{dls}{Dead Letter Sink}}]{DLS}{DLS}{Dead Letter Sink}

\newglossaryentry{csv}
{
  name=\glslink{CSV}{CSV},
  text= Comma-separated values,
  sort= csv,
  description= {It is a text file format that uses commas to separate values, and newlines to separate records. A CSV file stores tabular data (numbers and text) in plain text, where each line of the file typically represents one data record. Each record consists of the same number of fields, and these are separated by commas in the CSV file}
}
\newacronym[description={\glslink{csv}{Comma-separated values}}]{CSV}{CSV}{Comma-separated values}

\newglossaryentry{cdn}
{
  name=\glslink{CDN}{CDN},
  text= Content Delivery Network,
  sort= cdn,
  description= {It is a geographically distributed group of servers that caches content close to end users. A CDN allows for the quick transfer of assets needed for loading Internet content, including HTML pages, JavaScript files, stylesheets, images, and videos}
}
\newacronym[description={\glslink{cdn}{Content Delivery Network}}]{CDN}{CDN}{Content Delivery Network}

% --------------------------------------------------------
% No acronyms
% --------------------------------------------------------
\newglossaryentry{recaptcha}
{
  name=\glslink{recaptcha}{reCAPTCHA},
  text= reCAPTCHA,
  sort= recaptcha,
  description= {It is a free service from Google that helps protect websites from spam and abuse}
}

\newglossaryentry{job}
{
  name=\glslink{job}{Job},
  text= Job,
  sort= job,
  description= {It is a Kubernetes resource that creates one or more Pods and will continue to retry execution of the Pods until a specified number of them successfully terminate. As Pods successfully complete, the Job tracks the successful completions. When a specified number of successful completions is reached, the task (i.e. Job) is complete. Deleting a Job will clean up the Pods it created. Suspending a Job will delete its active Pods until the Job is resumed again}
}

\newglossaryentry{pod}
{
  name=\glslink{pod}{Pod},
  text= Pod,
  sort= pod,
  description= {It is the smallest deployable unit of computing that you can create and manage in Kubernetes. A Pod is a group of one or more containers, with shared storage and network resources, and a specification for how to run the containers}
}

\newglossaryentry{docker}
{
  name=\glslink{docker}{Docker},
  text= Docker,
  sort= docker,
  description= {It is an open platform for developing, shipping, and running applications. Docker enables you to separate your applications from your infrastructure so you can deliver software quickly}
}

\newglossaryentry{k8s}
{
  name=\glslink{k8s}{Kubernetes},
  text= Kubernetes,
  sort= k8s,
  description= {It is a portable, extensible, open-source platform for managing containerized workloads and services that facilitates both declarative configuration and automation}
}

\newglossaryentry{ns}
{
  name=\glslink{ns}{Namespace},
  text= Namespace,
  sort= ns,
  description= {It provides a mechanism for isolating groups of resources within a single cluster. Names of resources need to be unique within a namespace, but not across namespaces}
}

\newglossaryentry{svc}
{
  name=\glslink{svc}{Service},
  text= Service,
  sort= svc,
  description= {It is a method for exposing a network application that is running as one or more Pods in your cluster. The three most important service types are ClusterIP, NodePort and LoadBalancer}
}

\newglossaryentry{config_map}
{
  name=\glslink{config_map}{ConfigMap},
  text= ConfigMap,
  sort= config_map,
  description= {It is an API object used to store non-confidential data in key-value pairs. Pods can consume ConfigMaps as environment variables, command-line arguments, or as configuration files in a volume}
}

\newglossaryentry{secret}
{
  name=\glslink{secret}{Secret},
  text= Secret,
  sort= secret,
  description= {It is an object that contains a small amount of sensitive data such as a password, a token, or a key. Such information might otherwise be put in a Pod specification or in a container image.}
}

\newglossaryentry{deployment}
{
  name=\glslink{deployment}{Deployment},
  text= Deployment,
  sort= deployment,
  description= {It provides declarative updates for Pods and ReplicaSets}
}

\newglossaryentry{ingress}
{
  name=\glslink{ingress}{Ingress},
  text= Ingress,
  sort= ingress,
  description= {It is an API object that manages external access to the services in a cluster, typically HTTP. It may provide load balancing, SSL termination and name-based virtual hosting}
}

\newglossaryentry{route_53}
{
  name=\glslink{route_53}{Route 53},
  text= Route 53,
  sort= route_53,
  description= {It is a highly available and scalable DNS web service which allows domain registration, DNS routing, and health checking}
}

\newglossaryentry{cloudfront}
{
  name=\glslink{cloudfront}{Cloudfront},
  text= Cloudfront,
  sort= cloudfront,
  description= {It is a web service that speeds up distribution of your static and dynamic web content, such as .html, .css, .js, and image files, to your users}
}

\newglossaryentry{cheerio}
{
  name=\glslink{cheerio}{Cheerio},
  text= Cheerio,
  sort= cheerio,
  description= {It is a fast, flexible, and elegant library for parsing and manipulating HTML and XML}
}

\newglossaryentry{got_scraping}
{
  name=\glslink{got_scraping}{GotScraping},
  text= Got Scraping,
  sort= got_scraping,
  description= {It is a small but powerful \href{https://github.com/sindresorhus/got}{got} extension with the purpose of sending browser-like requests out of the box. This is very essential in the web scraping industry to blend in with the website traffic}
}

\newglossaryentry{domcontentloaded}
{
  name=\glslink{domcontentloaded}{DOMContentLoaded},
  text= DOMContentLoaded,
  sort= domcontentloaded,
  description= {It is an event fired when the HTML document has been fully parsed and all deferred scripts have been downloaded and executed. See \href{https://pptr.dev/api/puppeteer.puppeteerlifecycleevent}{PuppeteerLifeCycleEvent} for more details}
}

\newglossaryentry{hex_arch}
{
  name=\glslink{hex_arch}{Hexagonal Architecture},
  text= Hexagonal Architecture,
  sort= hex_arch,
  description= {It is an architectural pattern used in software design. It aims to create loosely coupled application components that can be easily connected to their software environment through ports and adapters. This makes components exchangeable at any level and facilitates test automation}
}

\newglossaryentry{no_sql}
{
  name=\glslink{no_sql}{NoSQL Database},
  text= NoSQL Database,
  sort= no_sql,
  description= {They are non-tabular databases and store data differently than relational tables. They come in a variety of types based on their data model. The main types are document, key-value, wide-column, and graph. They provide flexible schemas and scale easily with large amounts of data and high user loads}
}

\newglossaryentry{rabbit_mq}
{
  name=\glslink{rabbit_mq}{RabbitMQ},
  text= RabbitMQ,
  sort= rabbit_mq,
  description= {It is an open-source message-broker software that originally implemented the Advanced Message Queuing Protocol (AMQP) and has since been extended with a plug-in architecture to support Streaming Text Oriented Messaging Protocol (STOMP), MQ Telemetry Transport (MQTT), and other protocols}
}

\newglossaryentry{kafka}
{
  name=\glslink{kafka}{Apache Kafka},
  text= Apache Kafka,
  sort= kafka,
  description= {It is an open-source distributed event streaming platform used by thousands of companies for high-performance data pipelines, streaming analytics, data integration, and mission-critical applications.}
}

\newglossaryentry{circle_ci}
{
  name=\glslink{circle_ci}{CircleCI},
  text= CircleCI,
  sort= circle_ci,
  description= {It is a continuous integration and delivery platform that helps the development teams to release code rapidly and automate the build, test, and deploy}
}

\newglossaryentry{kustomize}
{
  name=\glslink{kustomize}{Kustomize},
  text= Kustomize,
  sort= kustomize,
  description= {It is a Kubernetes configuration transformation tool that enables you to customize untemplated YAML files, leaving the original files untouched}
}

\newglossaryentry{dynamo_db}
{
  name=\glslink{dynamo_db}{Dynamo DB},
  text= Dynamo DB,
  sort= dynamo_db,
  description= {It is a fully managed NoSQL database service that provides fast and predictable performance with seamless scalability}
}

\newglossaryentry{step_funcs}
{
  name=\glslink{step_funcs}{Step Functions},
  text= Step Functions,
  sort= step_funcs,
  description= {It is a serverless orchestration service that lets you integrate with AWS Lambda functions and other AWS services to build business-critical applications}
}

\newglossaryentry{kendra}
{
  name=\glslink{kendra}{Kendra},
  text= Kendra,
  sort= kendra,
  description= {It is an intelligent search service that uses natural language processing and advanced machine learning algorithms to return specific answers to search questions from your data}
}

\newglossaryentry{dockerfile}
{
  name=\glslink{dockerfile}{Dockerfile},
  text= Dockerfile,
  sort= dockerfile,
  description= {Docker can build images automatically by reading the instructions from a Dockerfile. It is a text document that contains all the commands a user could call on the command line to assemble an image}
}

\newglossaryentry{container}
{
  name=\glslink{container}{Container},
  text= Container,
  sort= container,
  description= {It is a standard unit of software that packages up code and all its dependencies so the application runs quickly and reliably from one computing environment to another}
}

\newglossaryentry{role}
{
  name=\glslink{role}{Role},
  text= Role,
  sort= role,
  description= {It is a collection of permissions that allow users to perform specific actions on a defined set of Kubernetes resource types}
}

\newglossaryentry{role_bind}
{
  name=\glslink{role_bind}{Role Binding},
  text= RoleBinding,
  sort= role_bind,
  description= {It grants the permissions defined by roles to the relevant user or user group}
}

\newglossaryentry{svc_account}
{
  name=\glslink{svc_account}{Service Account},
  text= ServiceAccount,
  sort= svc_account,
  description= {It provides an identity for processes that run in a Pod}
}

\newglossaryentry{side_car_container}
{
  name=\glslink{side_car_container}{Sidecar Container},
  text= Sidecar Container,
  sort= side_car_container,
  description= {It is the secondary container that runs along with the main application container within the same Pod. These containers are used to enhance or extend the functionality of the main application container by providing additional services or functionality, such as logging, monitoring, security, or data synchronization, without directly altering the primary application code}
}

\newglossaryentry{quorum_based_system}
{
  name=\glslink{quorum_based_system}{Quorum-based System},
  text= Quorum-based System,
  sort= quorum_based_system,
  description= {It is a mechanism by which decisions are made regarding data consistency and availability in a distributed environment. A quorum represents the minimum number of nodes that must agree on an operation to be successful. This ensures that data operations maintain consistency and resilience, even in the presence of node failures or network partitions}
}

\newglossaryentry{helm_chart}
{
  name=\glslink{helm_chart}{Helm Chart},
  text= Helm Chart,
  sort= helm_chart,
  description= {It allows you to manage Kubernetes manifests without using the Kubernetes \acrshort{CLI} or remembering complicated Kubernetes commands to control the cluster}
}